\documentclass[a4paper,12pt]{article} % This defines the style of your paper
%\documentclass[natbib]{svjour3}
\usepackage[top = 2cm, bottom = 2cm, left = 2cm, right = 2cm]{geometry} 
%\usepackage{titling}
\usepackage{amsmath}  
\usepackage{amsfonts}
 
\usepackage{algorithm}
\usepackage{algorithmic}
\usepackage{mathtools}
\usepackage[OT1]{fontenc}
\usepackage{pifont}
\usepackage{subcaption}
\usepackage{xcolor}
\usepackage{booktabs}
\usepackage{multirow}
\usepackage{graphics}
\newcommand{\tabincell}[2]{\begin{tabular}{@{}#1@{}}#2\end{tabular}} 
\newcommand*{\eg}{e.g., }
\newcommand*{\mycode}{\fontfamily{lmtt}\selectfont}
\newcommand*{\mynote}{\fontfamily{lmss}\selectfont}
\newcommand*{\ie}{i.e.,}
\newcommand*{\etc}{etc. }
\newcommand*{\all}{et al. }
\usepackage{tikz}
\newcommand*\circled[1]{\tikz[baseline=(char.base)]{
		\node[shape=circle,draw,inner sep=2pt] (char) {#1};}}
\newcommand{\mytodoblue}[1]{\textcolor{blue}{~#1}}
\newcommand{\mytodored}[1]{\textcolor{red}{~#1}}
\newcommand{\mytodogreen}[1]{\textcolor{mygreen}{~#1}}
\newcommand{\mytodoorange}[1]{\textcolor{orange}{~#1}}
\newcommand{\mytodocyan}[1]{\textcolor{cyan}{~#1}}
\newcommand{\mytodopink}[1]{\textcolor{purple}{~#1}}


\newcounter{numcomment}

\newcommand{\newhline}[0]{
	
	\
	
	\noindent\makebox[\linewidth]{\rule{\textwidth}{1pt}}
	
	\
	
}
\newcommand{\comment}[1]{\newhline\refstepcounter{numcomment}\textbf{Comment~\#\thenumcomment\label{#1}}:}
\newcommand{\brief}{\textbf{Brief Reponse~\#\thenumcomment}: }
\newcommand{\full}{\textbf{Full Reponse~\#\thenumcomment}: }
\newcommand{\response}{\textbf{Reponse~\#\thenumcomment}: }
%\newcommand*{\short}[1]{\textbf{Brief Reponse #{~#1}}:}
%\newcommand*{\full}[2]{\textbf{Full Reponse #{~#1}}: }


%\newenvironment{comment}[1][]{\refstepcounter{comment}\par\medskip
%	\textbf{Comment~\thecomment. #1} \rmfamily}{\medskip}


\DeclareRobustCommand{\legendsquare}[1]{%
	\textcolor{#1}{\rule{4ex}{2ex}}%
}
\newcommand{\scc}[1]{\mytodored{[scc: #1]}}
\newcommand{\yepang}[1]{\mytodoblue{[yepang: #1]}}
\newcommand{\civi}[1]{\mytodogreen{[civi: #1]}}
\newcommand{\victor}[1]{\mytodocyan{[victor: #1]}}
\newcommand{\ma}[1]{\mytodoorange{[ma: #1]}}


\title{Response Letter}
%\subtitle{EMSE manuscript #EMSE-D-20-00124}
\author{FirstName, LastName}
\date{May 16, 2021}
\begin{document}
	
	\maketitle
	\begin{center}
		\begin{tabular}{c} % This is a simple tabular environment to align your text nicely
			TOSEM/TSE/EMSE manuscript id: XXXXX \\
			Paper Title: First line \\ and the second line if necessary\\
		\end{tabular}
	\end{center}

\section{Overview}
In the overview, you should at least do the following things:

	First, please thank the reviewers and editors for their comments.
	
	Second, please highlight that you have address the all the major issues raised by the reviewers and editors, and almost all minor issues raised.
	
	Third, please summarize which significant revision is done in this version.
	
\section{Response to the Comments Raised by the Editor and by Multiple Reviewers}
\label{meta}
First, we are going to discuss the comments summarized by the editor and the comments from multiple reviewers.

\comment{q1}
%You can cite q1
Reviewer 1 and 3 both point out XXX. 

\brief
We reformulated the XXX.

\full
Now the XXX is changed as the following:
\textit{Your response here.}


\comment{}
Please explain what are the classic algorithms and provide references to them and to studies that have compared them.

\response
Your response here. 
The following is an example to refer the previous Response. 

Please note that more detailed can be found in the Response~\ref{q1}.




\section{Response to the Comments Raised by  Reviewer 1}

\comment{}
The acronym of the DL and DNN.

\response
To not confuse the readers, now the term ``DNN models'' is used in our paper, instead of using interchangeably it and `deep learning models`''. 


\bibliographystyle{acm}
\bibliography{example.bib}

\end{document}


