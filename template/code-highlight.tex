\documentclass{article}
\usepackage{url}
\usepackage{xspace}
\usepackage{xcolor}

\newcommand{\cf}{\hbox{\emph{cf.}}\xspace}
\newcommand{\deletia}{\ldots [deletia] \ldots}
\newcommand{\etal}{\hbox{\emph{et al.}}\xspace}
\newcommand{\eg}{\hbox{\emph{e.g.}}\xspace}
\newcommand{\ie}{\hbox{\emph{i.e.}}\xspace}
\newcommand{\scil}{\hbox{\emph{sc.}}\xspace} %scilicet: it is permitted to know
\newcommand{\st}{\hbox{\emph{s.t.}}\xspace}
\newcommand{\wrt}{\hbox{\emph{w.r.t.}}\xspace}
\newcommand{\etc}{\hbox{\emph{etc.}}\xspace}
\newcommand{\viz}{\hbox{\emph{viz.}}\xspace} %videlicet: it is permitted to see



% Comment
\newcommand{\cmt}[1]{\textit{\textcolor{red}{\footnotesize CMT: #1}}}
% Quote and comment.
\newcommand{\qcmt}[2]{\textit{\textcolor{red}{\footnotesize CMT: #1}} {\textcolor{blue}{#2}}}

% TODO..
\newcommand{\todo}[1]{\textit{\textcolor{red}{\footnotesize TODO: #1}}}
% Quote and todo.
\newcommand{\qtodo}[2]{\textit{\textcolor{red}{\footnotesize TODO: #1}} {\textcolor{blue}{#2}}}

% Leave comments with name
\definecolor{mygreen}{HTML}{02818a}
\newcommand{\mytodoblue}[1]{\textcolor{blue}{~#1}}
\newcommand{\mytodored}[1]{{\color{red}~#1}}
\newcommand{\mytodogreen}[1]{\textcolor{mygreen}{~#1}}
\newcommand{\mytodoorange}[1]{\textcolor{orange}{~#1}}
\newcommand{\mytodocyan}[1]{\textcolor{cyan}{~#1}}
\newcommand{\mytodopink}[1]{\textcolor{purple}{~#1}}

\DeclareRobustCommand{\legendsquare}[1]{%
	\textcolor{#1}{\rule{4ex}{2ex}}%
}
\mathchardef\mhyphen="2D


% Leave comments
\newcommand{\yiwen}[1]{\mytodoblue{[Yiwen:#1]}}
\newcommand{\cn}[1]{\mytodored{[cn: #1]}}
\newcommand{\victor}[1]{\mytodocyan{[Victor:#1]}}
\newcommand{\yourname}[1]{\mytodogreen{[XXX:#1]}}
\newcommand{\name}[1]{\mytodoorange{[name:#1]}}




\newcommand{\myparagraph}[1]{
  \vspace*{0.04cm}
  \noindent \textit{\textbf{#1.}}\quad
}
\newcommand{\mycode}[1]{\texttt{#1}\xspace}








\newcommand{\mt}[1]{%
    \tikz[overlay,remember picture,baseline] \coordinate (#1) at (0,0) {};}

\newcommand{\verticalline}{\unskip\ \vrule\ \ \ }

%the following should be copied into your preamble somewhere.
\definecolor{deepRed}{HTML}{D81B60}
\definecolor{deepBlue}{HTML}{1E88E5}
\definecolor{deepOrange}{HTML}{FFC107}

\newcommand{\highlightGreen}[2]{%
    \draw[deepBlue,line width=8pt,opacity=0.3]%
    ([yshift=2pt]#1) -- ([yshift=2pt]#2);%
}

\newcommand{\highlightRed}[2]{%
    \draw[deepRed,line width=8pt,opacity=0.3]%
    ([yshift=2pt]#1) -- ([yshift=2pt]#2);%
}

\newcommand{\highlightOrange}[2]{%
    \draw[deepOrange,line width=8pt,opacity=0.3]%
    ([yshift=2pt]#1) -- ([yshift=2pt]#2);%
}

\newcommand{\highlightGray}[2]{%
    \draw[gray,line width=8pt,opacity=0.7]%
    ([yshift=2pt]#1) -- ([yshift=2pt]#2);%
}

\newcommand{\highlight}[2][yellow]{%
    \tikz[baseline=(X.base)]%
    \node[fill=#1!30, rounded corners=0pt, inner sep=1pt, outer sep=0pt](X){#2};%
}

\newcommand{\highlightnarrow}[2]{%
    \tikz[baseline=(X.base), overlay, remember picture]%
    \node[anchor=base] (X) {#2};
    \draw[fill=#1!30, rounded corners=0pt, inner sep=0pt] ([yshift=2pt,xshift=-1pt]X.north west) -- ([yshift=2pt,xshift=1pt]X.north east) -- ([yshift=-1pt,xshift=1pt]X.south east) -- ([yshift=-1pt,xshift=-1pt]X.south west) -- cycle;%
}


\usepackage{cleveref} % This package must be loaded in the end.

\Crefname{table}{Table}{Tables}
\crefname{table}{Table}{Tables}
\Crefname{figure}{Figure}{Figures}
\crefname{figure}{Figure}{Figures}
\Crefname{algocf}{Algorithm}{Algorithms}
\crefname{algocf}{Algorithm}{Algorithms}
\Crefname{algorithm}{Algorithm}{Algorithms}
\crefname{algorithm}{Algorithm}{Algorithms}
\crefname{thm}{Theorem}{Theorems}
\Crefname{thm}{Theorem}{Theorems}

% https://tex.stackexchange.com/questions/81634/cleveref-configure-symbol-for-all-sectioning-types-once-time
\crefformat{chapter}{\S#2#1#3}
\crefmultiformat{chapter}{\S\S#2#1#3}{ and~#2#1#3}{, #2#1#3}{, and~#2#1#3}

\crefformat{section}{\S#2#1#3}
\crefmultiformat{section}{\S\S#2#1#3}{ and~#2#1#3}{, #2#1#3}{, and~#2#1#3}


\begin{document}
\begin{figure*}[ht]
    \centering
    \begin{subfigure}[b]{0.50\linewidth}
        %    \begin{minipage}{.23\linewidth}
            \begin{lstlisting}[language=C,  escapechar=@, numbers=left,
                numbersep=1em, commentstyle=\color{darkgray},
                xleftmargin=1em, tabsize=1]
#include <stdint.h>
uint8_t g;
int16_t div(int16_t a, int16_t b) {
    return b == 0 ? a : a / b;
}
int64_t id(a) { return a; }
int8_t fn(a) {
    uint8_t b;
    for (;;) {
        uint16_t c = 65532UL;
        id(b |= g);
        if (div(c,
        div(0x2FAAL, 65535UL)))
        @\mt{1s}@b |= g;@\mt{1e}@ @\label{fig:firstline}@ @\label{subfig:motivation:our_variant:assignment1}@
        @\mt{2s}@else {@\mt{2e}@
            a &= 1L;
            @\mt{3s}@}@\mt{3e}@
        @\mt{4s}@b |= g;@\mt{4e}@ @\label{line:secondline}@ @\label{subfig:motivation:our_variant:assignment2}@
    }
}
int main() { fn(0); }
            \end{lstlisting}
            %    \end{minipage}
        \caption{\emph{Variant} by .}
        \label{subfig:motivation:our_variant}
    \end{subfigure}
    \hfil
    \verticalline
    \begin{subfigure}[b]{0.45\linewidth}
\begin{lstlisting}[language=C,  escapechar=@, numbers=none,
    numbersep=1em, commentstyle=\color{darkgray},
    xleftmargin=1em, tabsize=1]
    ......
    int8_t fn() {
        ......
    }
    ......
    int main() {
        ......
        @\mt{5s}@fn@\mt{5e}@;
    }
\end{lstlisting}
\caption{\emph{Seed} by .}
\label{subfig:motivation:dd_seed}
%
%
\vspace{1ex}
%
\begin{lstlisting}[language=C,  escapechar=@, numbers=none,
    numbersep=1em, commentstyle=\color{darkgray},
    xleftmargin=1em, tabsize=1]
    ......
    int8_t fn() {
        ......
    }
    ......
    int main() {
        ......
        @\mt{6s}@fn()@\mt{6e}@;
% this places two tags, 6s and 6e
% note that the command mt
% needs to be escapted
    }
\end{lstlisting}
\caption{\emph{Variant} by .}
\label{subfig:motivation:dd_variant}
\end{subfigure}


\begin{tikzpicture}[remember picture, overlay]
    \highlightGreen{1s}{1e}
%    the above highlight the code between 1s and 1e.
%    cannot across lines
    \highlightGreen{2s}{2e}
    \highlightGreen{3s}{3e}
    \highlightGreen{4s}{4e}
    \highlightGreen{5s}{5e}
    \highlightGreen{6s}{6e}
\end{tikzpicture}
    \caption{
        XXX are highlighted in \highlight[deepBlue]{blue}.
        Read the comments to understand how this works.
}
\label{fig:code_llvm_21467_motivation_developer}

\end{figure*}
\end{document}
