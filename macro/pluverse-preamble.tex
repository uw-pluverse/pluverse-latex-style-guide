\usepackage[ruled,lined,linesnumbered,vlined]{algorithm2e}
\newcommand\mycommfont[1]{\footnotesize\textcolor{black!60}{#1}}
\SetCommentSty{mycommfont}

\usepackage{amsmath}
\usepackage{amsfonts}
\usepackage{amsthm}
\usepackage{balance}
\usepackage{enumitem}
% The following packages will lead to a format issue in ACM template (i.e., wrong font size of the title)
% \usepackage[T1]{fontenc}
% \usepackage[scaled]{helvet}
\usepackage{graphicx}
\usepackage{listings}
\usepackage{multicol}
\usepackage{multirow}
\usepackage{natbib}
\usepackage{subcaption}
\usepackage{url}
\usepackage{xspace}
\usepackage{xcolor}
\usepackage{tcolorbox}
\usepackage{colortbl}

\lstset{
  language=C,
  tabsize=1,
  %line numbers....
  escapeinside={(*@}{@*)},
  basicstyle=\footnotesize\ttfamily,
  stringstyle=\color{blue},
  keywordstyle=\color{blue}\bfseries,
  commentstyle=\small\color{cyan}\bfseries,
  %directivestyle=\color{blue},
  %frame=shadowbox,
  xleftmargin=0em,
  xrightmargin=0em,
}

\usepackage{tikz}
\usepackage{pgfplots}
\newcommand*\circled[1]{\tikz[baseline=(char.base)]{
    \node[shape=circle,draw,inner sep=0.5pt] (char) {\small#1};}}
\usetikzlibrary{shapes}
\usetikzlibrary{shapes.geometric}
\usetikzlibrary{arrows.meta, positioning}


\newcommand{\aka}{\hbox{\emph{a.k.a.}}\xspace}
\newcommand{\cf}{\hbox{\emph{cf.}}\xspace}
\newcommand{\deletia}{\ldots [deletia] \ldots}
\newcommand{\eg}{\hbox{\emph{e.g.}}\xspace}
\newcommand{\etal}{\hbox{\emph{et al.}}\xspace}
\newcommand{\etc}{\hbox{\emph{etc.}}\xspace}
\newcommand{\ie}{\hbox{\emph{i.e.}}\xspace}
\newcommand{\scil}{\hbox{\emph{sc.}}\xspace} %scilicet: it is permitted to know
\newcommand{\st}{\hbox{\emph{s.t.}}\xspace}
\newcommand{\viz}{\hbox{\emph{viz.}}\xspace} %videlicet: it is permitted to see
\newcommand{\vs}{\hbox{\emph{vs.}}\xspace}
\newcommand{\wrt}{\hbox{\emph{w.r.t.}}\xspace}



% Comment
\newcommand{\cmt}[1]{\textit{\textcolor{red}{\footnotesize CMT: #1}}}
% Quote and comment.
\newcommand{\qcmt}[2]{\textit{\textcolor{red}{\footnotesize CMT: #1}} {\textcolor{blue}{#2}}}

% TODO..
\newcommand{\todo}[1]{\textit{\textcolor{red}{\footnotesize TODO: #1}}}
% Quote and todo.
\newcommand{\qtodo}[2]{\textit{\textcolor{red}{\footnotesize TODO: #1}} {\textcolor{blue}{#2}}}

% Leave comments with name
\definecolor{mygreen}{HTML}{02818a}
\newcommand{\mytodoblue}[1]{\textcolor{blue}{~#1}}
\newcommand{\mytodored}[1]{{\color{red}~#1}}
\newcommand{\mytodogreen}[1]{\textcolor{mygreen}{~#1}}
\newcommand{\mytodoorange}[1]{\textcolor{orange}{~#1}}
\newcommand{\mytodocyan}[1]{\textcolor{cyan}{~#1}}
\newcommand{\mytodopink}[1]{\textcolor{purple}{~#1}}

\DeclareRobustCommand{\legendsquare}[1]{%
	\textcolor{#1}{\rule{4ex}{2ex}}%
}
\mathchardef\mhyphen="2D


% Leave comments
\newcommand{\cn}[1]{\mytodored{[cn: #1]}}
\newcommand{\qcn}[2]{\cn{#1}{\textcolor{violet}{~#2}}}

\newcommand{\yiwen}[1]{\mytodoblue{[Yiwen:#1]}}

\newcommand{\victor}[1]{\mytodocyan{[Victor:#1]}}
\newcommand{\qvictor}[2]{\victor{#1}{\textcolor{violet}{~#2}}}

\newcommand{\zy}[1]{\mytodogreen{[XXX:#1]}}
\newcommand{\mx}[1]{\mytodoorange{[Max:#1]}}


\newcounter{FindingCounter}
\newcommand{\finding}[1]{
	\begin{tcolorbox}
		\textbf{RQ\refstepcounter{FindingCounter}\theFindingCounter}: #1
	\end{tcolorbox}
}



\newcommand{\myparagraph}[1]{
  \vspace*{0.04cm}
  \noindent \textit{\textbf{#1.}}\quad
}

\newcommand{\mycode}[1]{\texttt{#1}\xspace}

\newcommand{\projectname}[1]{\textsf{#1}\xspace}




