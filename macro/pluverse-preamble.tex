\usepackage[ruled,lined,linesnumbered,vlined]{algorithm2e}
\newcommand\mycommfont[1]{\footnotesize\textcolor{black!60}{#1}}
\SetCommentSty{mycommfont}
\DontPrintSemicolon % Do not show the semicolons.

\usepackage{amsmath}
\usepackage{amsfonts}
\usepackage{amsthm}
\usepackage{balance}
\usepackage{enumitem}
% The following packages will lead to a format issue in ACM template (i.e., wrong font size of the title)
% \usepackage[T1]{fontenc}
% \usepackage[scaled]{helvet}
\usepackage{graphicx}
\usepackage{listings}
\usepackage{multicol}
\usepackage{multirow}
\usepackage{natbib}
\usepackage{subcaption}
\usepackage{url}
\usepackage{xspace}
\usepackage{xcolor}
\usepackage{tcolorbox}
\usepackage{colortbl}

\lstset{
  language=C,
  tabsize=1,
  %line numbers....
  escapeinside={(*@}{@*)},
  basicstyle=\footnotesize\ttfamily,
  stringstyle=\color{blue},
  keywordstyle=\color{blue}\bfseries,
  commentstyle=\small\color{cyan}\bfseries,
  %directivestyle=\color{blue},
  %frame=shadowbox,
  xleftmargin=0em,
  xrightmargin=0em,
}

\usepackage{tikz}
\usepackage{pgfplots}
\newcommand*\circled[1]{\tikz[baseline=(char.base)]{
    \node[shape=circle,draw,inner sep=0.5pt] (char) {\small#1};}}
\usetikzlibrary{shapes}
\usetikzlibrary{shapes.geometric}
\usetikzlibrary{arrows.meta, positioning}


\newcommand{\aka}{\mbox{a.k.a.}\xspace}
\newcommand{\cf}{\mbox{cf.}\xspace}
\newcommand{\deletia}{\ldots [deletia] \ldots}
\newcommand{\eg}{\mbox{e.g.}\xspace}
\newcommand{\etal}{\mbox{et al.}\xspace}
\newcommand{\etc}{\mbox{etc.}\xspace}
\newcommand{\ie}{\mbox{i.e.}\xspace}
\newcommand{\scil}{\mbox{sc.}\xspace} %scilicet: it is permitted to know
\newcommand{\st}{\mbox{s.t.}\xspace}
\newcommand{\viz}{\mbox{viz.}\xspace} %videlicet: it is permitted to see
\newcommand{\vs}{\mbox{vs.}\xspace}
\newcommand{\wrt}{\mbox{w.r.t.}\xspace}


% Coloring for ColorBlindness
% https://davidmathlogic.com/colorblind/#%23332288-%23117733-%2344AA99-%2388CCEE-%23DDCC77-%23CC6677-%23AA4499-%23882255
% Here are the colors from the Paul Tol color palette.
\definecolor{BlindColorTolOne}{HTML}{332288}
\definecolor{BlindColorTolTwo}{HTML}{117733} % green
\definecolor{BlindColorTolThree}{HTML}{44AA99}
\definecolor{BlindColorTolFour}{HTML}{88CCEE}
\definecolor{BlindColorTolFive}{HTML}{DDCC77}
\definecolor{BlindColorTolSix}{HTML}{CC6677} % light red
\definecolor{BlindColorTolSeven}{HTML}{AA4499}
\definecolor{BlindColorTolEight}{HTML}{882255}



%https://davidmathlogic.com/colorblind/#%23000000-%23E69F00-%2356B4E9-%23009E73-%23F0E442-%230072B2-%23D55E00-%23CC79A7
\definecolor{BlindColorWongOne}{HTML}{000000} % black
\definecolor{BlindColorWongTwo}{HTML}{E69F00}
\definecolor{BlindColorWongThree}{HTML}{56B4E9}
\definecolor{BlindColorWongFour}{HTML}{009E73}
\definecolor{BlindColorWongFive}{HTML}{F0E442}
\definecolor{BlindColorWongSix}{HTML}{0072B2} % blue
\definecolor{BlindColorWongSeven}{HTML}{D55E00}
\definecolor{BlindColorWongEight}{HTML}{CC79A7}
% End of Coloring for ColorBlindness

% Comment
\newcommand{\cmt}[1]{\textit{\textcolor{BlindColorTolEight}{\footnotesize CMT: #1}}}
% Quote and comment.
\newcommand{\qcmt}[2]{\textit{\textcolor{BlineColorTolEight}{\footnotesize CMT: #1}} {\textcolor{blue}{#2}}}

% TODO..
\newcommand{\todo}[1]{\textit{\textcolor{BlineColorTolEight}{\footnotesize TODO: #1}}}
% Quote and todo.
\newcommand{\qtodo}[2]{\textit{\textcolor{BlineColorTolEight}{\footnotesize TODO: #1}} {\textcolor{blue}{#2}}}

% Leave comments with name
\definecolor{mygreen}{HTML}{02818a}
\newcommand{\mytodoblue}[1]{\textcolor{blue}{~#1}}
\newcommand{\mytodored}[1]{{\color{red}~#1}}
\newcommand{\mytodogreen}[1]{\textcolor{mygreen}{~#1}}
\newcommand{\mytodoorange}[1]{\textcolor{orange}{~#1}}
\newcommand{\mytodocyan}[1]{\textcolor{cyan}{~#1}}
\newcommand{\mytodopink}[1]{\textcolor{purple}{~#1}}

\DeclareRobustCommand{\legendsquare}[1]{%
	\textcolor{#1}{\rule{4ex}{2ex}}%
}
\mathchardef\mhyphen="2D

% Leave comments
\newcommand{\NamedCommentTemplate}[3]{\colorbox{#1}{\textcolor{white}{\scriptsize #2:}}~\textcolor{#1}{$\lceil$#3$\rfloor$}}
\newcommand{\cn}[1]{\NamedCommentTemplate{BlindColorTolSeven}{CN}{#1}}
\newcommand{\qcn}[2]{\cn{#1}{\textcolor{violet}{~#2}}}
\newcommand{\yiwen}[1]{\NamedCommentTemplate{BlindColorTolOne}{Yiwen}{#1}}
\newcommand{\victor}[1]{\NamedCommentTemplate{BlindColorTolTwo}{Victor}{#1}}
\newcommand{\qvictor}[2]{\victor{#1}{\textcolor{violet}{~#2}}}
\newcommand{\zy}[1]{\NamedCommentTemplate{BlindColorTolThree}{Zhenyang}{#1}}
\newcommand{\mx}[1]{\NamedCommentTemplate{BlindColorTolSix}{Max}{#1}}
\newcommand{\yr}[1]{\NamedCommentTemplate{BlindColorTolFive}{Yiran}{#1}}

\newcommand{\verticalline}{\unskip\ \vrule\ \ \ }
\newcommand{\reducedstrut}{\vrule width 0pt height .9\ht\strutbox depth .9\dp\strutbox\relax}
\newcommand{\LeanColorBox}[2]{
	\begingroup
	\unskip % It seems that the colorbox introduces an extra whitespace before the text. We need to delete the space.
	\setlength{\fboxsep}{0pt}%
	\colorbox{#1}{\reducedstrut#2\/}%
	\endgroup
}

\newcounter{FindingCounter}
\newcommand{\finding}[1]{
	\begin{tcolorbox}
		\textbf{RQ\refstepcounter{FindingCounter}\theFindingCounter}: #1
	\end{tcolorbox}
}



\newcommand{\myparagraph}[1]{
  \vspace*{0.04cm}
  \noindent \textit{\textbf{#1.}}\quad
}

\newcommand{\mycode}[1]{\texttt{#1}\xspace}

\newcommand{\projectname}[1]{\textsf{\mbox{#1}}} % \xspace is not needed here.
\SetKwData{AlgTrue}{true}
\SetKwData{AlgFalse}{false}



